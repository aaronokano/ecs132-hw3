\documentclass{article}
\usepackage{amsmath, amsthm, amssymb}
\usepackage{listings}
\usepackage{graphicx}
\usepackage{float}
\usepackage{enumerate}
\usepackage{fancyhdr}
\usepackage[labelfont=bf]{caption}
\usepackage[left=0.75in, top=1in, right=0.75in, bottom=1in]{geometry}
\pagestyle{plain}
\begin{document}
\rhead{Aaron Okano, Anatoly Torchinsky, Samuel Huang, Justin Maple \\ 
      ECS 132: Homework 3}
\thispagestyle{fancy}

% Let the homework begin!
\section*{Problem 1}

First, we observe that $L$ can be transformed into a normally distributed
random variable with mean 0 and standard deviation 1. We do this by creating
a new random variable $Z = \frac{L - \mu}{\sigma}$. The probability $P( L > 12
)$ can now be expressed as $P( Z > \frac{12 - 0}{\sigma} )$. We can use the R
function qnorm to find the value $c$ for which $P( Z < c ) = 1 - p$ where $p$
is 2.5\%, as given in the problem. Thus, calling \texttt{qnorm( 1 - 0.025 )}
gives us 1.96 as the value for $c$ and solving for $\sigma$ in the equation $c
= \frac{12}{\sigma}$ yields $\sigma \approx 6.123$. We can now plug this value
into \verb pnorm()  to find $P( L > 12 )$ when $\sigma$ increases by 10\%:
\begin{verbatim}
1 - pnorm( 12, 0, 1.1 * ( 12 / qnorm( 0.975 ) ) )
\end{verbatim}
This produces a value of approximately 3.74\%, so flooding increased from 2.5\%
of days to 3.74\% of days.

\section*{Problem 2}

$P( X > 3.0 )$ can be easily derived by integrating the density function over
all $X > 3.0$.
\begin{align*}
  P( X > 3.0 ) &= \int_{3}^{\infty} \frac{1}{t^2} dt \\
               &= \left.-\frac{1}{t} \right|_3^\infty \\
               &= \frac{1}{3}
\end{align*}
To find $E(X^{0.5})$, we need to multiply $t^{0.5}$ by the probability for each
$t$ and sum them all together (i.e.\ integrate).
\begin{align*}
  E( X^{0.5} ) &= \int_1^\infty t^{0.5}\cdot\frac{1}{t^2} dt \\
  &= \int_1^\infty \frac{1}{t^{1.5}} dt \\
  &=\left. - \frac{2}{t^{0.5}} \right|_1^\infty \\
  &= 2
\end{align*}
Simulation verifies both those calculations. To perform the simulation, we needed to
find a way to generate random numbers from the distribution. From the book, we
know that we can take the inverse of the cdf and then substitute random
numbers from the uniform distribution for $t$. First, we find the cdf:
\begin{align*}
  F_X(t) &= \int_1^t \frac{1}{t^2} \ dt \\
         &= \left. - \frac{1}{t} \right|_1^t \\
         &= 1 - \frac{1}{t}
\end{align*}
Then, the inverse can be found by swapping $F_X$ and $t$, substituting
$F_X^{-1}$ for $F_X$, and solving for $F_X^{-1}$.
\begin{align*}
  t &= 1 - \frac{1}{F_X^{-1}} \\
  \Rightarrow F_X^{-1} &= \frac{1}{1 - t}
\end{align*}

\section*{Problem 3}

There were two steps involved in proving $Cov(V) = c^{2}(A'A)^{-1}$.

First, simplify $V$:

\begin{align*}
  V &= (A'A)^{-1}A'U \\
    &= A^{-1}A'^{-1}A'U \\
    &= A^{-1}U
\end{align*}

Next, plug the simplified $V$ into the $Cov(V)$ formula:

\begin{align*}
  Cov(V) &= Cov(A^{-1}U) \\
         &= A^{-1}Cov(U)A'^{-1} \\
         &= A^{-1}(c^{2}I)A'^{-1} \\
         &= c^{2}A^{-1}IA'^{-1} \\
         &= c^{2}A^{-1}A'^{-1} \\
         &= c^{2}(A'A)^{-1}
\end{align*}

\end{document}
